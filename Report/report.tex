\documentclass[12pt]{article}

\usepackage{fullpage}

\usepackage{geometry}
\geometry{a4paper}

\usepackage{graphicx}
\linespread{1.2}
\usepackage{lmodern}
\usepackage{ragged2e}
\usepackage{amsmath}


\begin{document}
\begin{titlepage}

\newcommand{\HRule}{\rule{\linewidth}{0.5 mm}}
\center
\textsc{}\\[1cm]
\textsc{\Large Ict for Smart Societies}\\[0.4cm]
\textsc{\large Politecnico di Torino}\\[0.4cm]
\HRule\\[0.5cm]
{\huge \texttt{ICT FOR HEALTH LABORATORY}}\\[0.05cm]
\HRule\\[3cm]

\begin{minipage}{\textwidth}
\large
\emph{Authors:}\\[0.4cm]
\begin{tabular}{lll}
\textsc{Putina} &Andrian & s226673\\
\end{tabular}
\end{minipage}

\vspace{0.7cm}

\begin{minipage}{\textwidth}
\large
\emph{Professor:}\\[0.2cm]
\begin{tabular}{ll}
\textsc{Visentin} &Monica
\end{tabular}
\end{minipage}

\vspace{1cm}

\center {\includegraphics[scale=0.5]{logo.png}}\\[1cm]

{\large \today}\\[3cm]

\vfill

\end{titlepage}

%------------------------FINE PRIMA PAGINA---------------------------------------

%------------------------SECONDA PAG---------------------------
\tableofcontents
\newpage
%--------------------------------------------------------------

\section{Lab1: Linear Regression \& Parkinson}
The aim of the laboratory is understanding the basic idea of the regression and apply different techniques to obtain the desired results. In particular, will be considered a database containing a real Parkinson's Disease dataset and different regression solutions will by applied:
\begin{itemize}
\item MSE (Minimum Square Error)
\item Gradient Algorithm
\item Steepest Descent Algorithm
\end{itemize} 
Source: \footnotesize{ https://archive.ics.uci.edu/ml/machine-learning-databases/parkinsons/telemonitoring/}







\end{document}